\documentclass[12pt]{article}
\setlength{\oddsidemargin}{0in}
\setlength{\evensidemargin}{0in}
\setlength{\textwidth}{6.5in}
\setlength{\parindent}{0in}
\setlength{\parskip}{\baselineskip}
\usepackage{amsmath,amsfonts,amssymb}
\usepackage{graphicx}
\usepackage{enumitem}
\usepackage[]{algorithmicx}
\usepackage{amsthm}
\usepackage{fancyhdr}
\pagestyle{fancy}
\setlength{\headsep}{36pt}
\usepackage{tkz-berge}
\usetikzlibrary{positioning, automata}

\usepackage{hyperref}

\theoremstyle{remark}
\newtheorem*{solution}{Solution}

\newcommand{\makenonemptybox}[2]{%
%\par\nobreak\vspace{\ht\strutbox}\noindent
\item[]
\fbox{% added -2\fboxrule to specified width to avoid overfull hboxes
% and removed the -2\fboxsep from height specification (image not updated)
% because in MWE 2cm is should be height of contents excluding sep and frame
\parbox[c][#1][t]{\dimexpr\linewidth-2\fboxsep-2\fboxrule}{
  \hrule width \hsize height 0pt
  #2
 }%
}%
\par\vspace{\ht\strutbox}
}
\makeatother

\begin{document}

\lhead{{\bf CSCI 3104, Algorithms \\ Problem Set 8b (18 \textbf{Extra credit }points)} }
\rhead{Name: Luna McBride \\ ID: 107607144 \\ {\bf Profs.\ Hoenigman \& Agrawal\\ Fall 2019, CU-Boulder}}
\renewcommand{\headrulewidth}{0.5pt}

\phantom{Test}

\begin{small}
\textit{Advice 1}:\ For every problem in this class, you must justify your answer:\ show how you arrived at it and why it is correct. If there are assumptions you need to make along the way, state those clearly.
\vspace{-3mm} 

\textit{Advice 2}:\ Verbal reasoning is typically insufficient for full credit. Instead, write a logical argument, in the style of a mathematical proof.\\
\vspace{-3mm} 

\textbf{Instructions for submitting your solution}:
\vspace{-5mm} 

\begin{itemize}
	\item The solutions \textbf{should be typed} and we cannot accept hand-written solutions. \href{http://ece.uprm.edu/~caceros/latex/introduction.pdf}{Here's a short intro to Latex.}
	\item You should submit your work through \href{https://www.gradescope.com/courses/59294}{\textbf{Gradescope}} only.
	\item If you don't have an account on it, sign up for one using your CU email. You should have gotten an email to sign up. If your name based CU email doesn't work, try the identikey@colorado.edu version. 
	\item Gradescope will only accept \textbf{.pdf} files (except for code files that should be submitted separately on Gradescope if a problem set has them) and \textbf{try to fit your work in the box provided}. 
	\item You cannot submit a pdf which has less pages than what we provided you as Gradescope won't allow it. 
\end{itemize}
\vspace{-4mm} 
\end{small}

\hrulefill
\pagebreak
\\
\textbf{Important:} 

\begin{enumerate}
    \item This assignment has 1 (Q1) coding question. 
\begin{itemize}
    \item You need to submit 1 python file.
    \item The .py file should run for you to get points and name the file as following - \\
If Q1 asks for a python code, please submit it with the following naming convention - \texttt{Lastname-Firstname-PS8b-Q1.py}.
\item You need to submit the code via Canvas but the table/plot/result should be on the main .pdf.
\end{itemize}

\item This assignment is all Extra Credit.
\item \textbf{This doesn't have any late submissions.}
\end{enumerate}


\pagebreak

\begin{enumerate}



\item (18 pts total) In recitation, you learned about about $Quickselect$ algorithm, which can find the $k^{th}$ order statistic (or the $k^{th}$ smallest number) of a given unsorted array in $O(n)$ average time. You will implement $QuickSelect$ and use it to find a $Pivot$ that gives you a desirable split every time you perform $Partition$ in $QuickSort$.

\begin{itemize}
\item Implement $QuickSelect(A, p, r, split)$ \textbf{to find a suitable pivot} at each round such that it splits the subarray in consideration in two parts of size roughly 1/4 and 3/4 after partition (or 0.25-0.75 split). This implementation doesn't need the order statistic $k$ but only the smaller split fraction which in this case is $split = 0.25$. \textbf{'split' is not the order statistic 'k'}.
\item Swap and put this pivot (which will achieve the desired 1/4-3/4 split) at the end and perform $QuickSort(A, p, r)$ as usual. 
\item Count the total number of comparisons in one run of the QuickSort. Note that the number of comparisons in QuickSort is proportional to the runtime of Quicksort. This will help you verify the correctness of your plot.
\item Your implementation should be in-place (should not use an extra array to do QuickSort).
\item There will be some points associated with comments and good programming practices for code and result presentation.
\item Code for the experiment should also be present to get any points for the results.
\end{itemize}

\begin{enumerate}
\item (16 pts) Write a single Python code, which takes as input a positive integer n, \textbf{randomly shuffles an array of size n} with elements $[1, \ldots, n]$ and performs QuickSort on that shuffled array following the above conditions and counts the total comparisons.
\\\textbf{Experiment -} Run your code on a bunch of n values from $[2, 2^2, 2^3, .... 2^{10}]$ and present your result in a table with one column as the value of n and another as the number of comparisons. Alternatively, you can present your table in form of a labeled plot with the 2 columns forming the 2 axes.\\ Also, print the shuffled array and the sorted output for $n = 5, 10, 20$ showing the state of the array $A$ after each recursive call for each of these three $n$ values. \\
(All the result and prints should be on the pdf and the code should be submitted on Canvas.)
\item (2 pts) Assume the average runtime for QuickSelect i.e. $O(n)$, write the recurrence relation for your QuickSort implementation that selects pivot using your QuickSelect (include lower order terms for the sake of clarity). Explain the recurrence briefly. \\
\begin{solution}
$\newline$ Possible relations: $\newline$ Best Case: T(n)=2T($\frac{n}{2}$)+n $\newline$ Worst Case: T(n)=2T(n-1)+n $\newline \newline$ As it is, since quickselect is $O(n)$, it does not increase the overall cases that QuickSort gives. In terms of my code, the worst case always came up just a bit over $n^2$. This is because although the worst case is $n^2$ for a QuickSort, Big O does not account for the smaller terms that make it that much bigger (in a relatively small way, of course). However, because the jump above $n^2$ is so small relative to each individual quickselect that is done, it does relatively nothing to the overall count, and as such, worst case remains at $n^2$, and as such, remains at the speed of a QuickSort in the grand scheme of the program.
\end{solution}
\end{enumerate}


\end{enumerate}
\pagebreak
\begin{solution}
$\newline$ Outputs for n=5,10,20: $\newline$
[5 1 3 4 2] $\newline$
[1 2 3 4 5]$\newline$
[1 2 3 4 5]$\newline$
[1 2 3 4 5]$\newline$
[1 2 3 4 5]$\newline$
[10  7  3  1  4  5  6  9  8  2]$\newline$
[ 1  2  3  4  5  6  7  8  9 10]$\newline$
[ 1  2  3  4  5  6  7  8  9 10]$\newline$
[ 1  2  3  4  5  6  7  8  9 10]$\newline$
[ 1  2  3  4  5  6  7  8  9 10]$\newline$
[ 1  2  3  4  5  6  7  8  9 10]$\newline$
[ 1  2  3  4  5  6  7  8  9 10]$\newline$
[ 1  2  3  4  5  6  7  8  9 10]$\newline$
[ 1  2  3  4  5  6  7  8  9 10]$\newline$
[ 1  2  3  4  5  6  7  8  9 10]$\newline$
[ 1  2  3  4  5  6  7  8  9 10]$\newline$
[ 8  6 12  3  2 17 11  5 10 15 13 14  7  9  1 20 16 18  4 19]$\newline$
[ 1  2  3  4  5  6  7  8  9 10 11 12 13 14 15 16 17 18 19 20]$\newline$
[ 1  2  3  4  5  6  7  8  9 10 11 12 13 14 15 16 17 18 19 20]$\newline$
[ 1  2  3  4  5  6  7  8  9 10 11 12 13 14 15 16 17 18 19 20]$\newline$
[ 1  2  3  4  5  6  7  8  9 10 11 12 13 14 15 16 17 18 19 20]$\newline$
[ 1  2  3  4  5  6  7  8  9 10 11 12 13 14 15 16 17 18 19 20]$\newline$
[ 1  2  3  4  5  6  7  8  9 10 11 12 13 14 15 16 17 18 19 20]$\newline$
[ 1  2  3  4  5  6  7  8  9 10 11 12 13 14 15 16 17 18 19 20]$\newline$
[ 1  2  3  4  5  6  7  8  9 10 11 12 13 14 15 16 17 18 19 20]$\newline$
[ 1  2  3  4  5  6  7  8  9 10 11 12 13 14 15 16 17 18 19 20]$\newline$
[ 1  2  3  4  5  6  7  8  9 10 11 12 13 14 15 16 17 18 19 20]$\newline$
[ 1  2  3  4  5  6  7  8  9 10 11 12 13 14 15 16 17 18 19 20]$\newline$
[ 1  2  3  4  5  6  7  8  9 10 11 12 13 14 15 16 17 18 19 20]$\newline$
[ 1  2  3  4  5  6  7  8  9 10 11 12 13 14 15 16 17 18 19 20]$\newline$
[ 1  2  3  4  5  6  7  8  9 10 11 12 13 14 15 16 17 18 19 20]$\newline$
[ 1  2  3  4  5  6  7  8  9 10 11 12 13 14 15 16 17 18 19 20]$\newline$
[ 1  2  3  4  5  6  7  8  9 10 11 12 13 14 15 16 17 18 19 20]$\newline$
[ 1  2  3  4  5  6  7  8  9 10 11 12 13 14 15 16 17 18 19 20]$\newline$
[ 1  2  3  4  5  6  7  8  9 10 11 12 13 14 15 16 17 18 19 20]$\newline$
[ 1  2  3  4  5  6  7  8  9 10 11 12 13 14 15 16 17 18 19 20]$\newline$
[ 1  2  3  4  5  6  7  8  9 10 11 12 13 14 15 16 17 18 19 20]$\newline$
$\newline \newline$
 N   Conditionals$\newline$
---- ------------$\newline$
   2            2$\newline$
   4           20$\newline$
   8           44$\newline$
  16          169$\newline$
  32          578$\newline$
  64         1352$\newline$
 128        13793$\newline$
 256        63456$\newline$
 512       243304$\newline$
1024       877421$\newline$
\end{solution}

\pagebreak
(Extra space)

\pagebreak
(Extra space)

\pagebreak
(Extra space)

\end{document}


